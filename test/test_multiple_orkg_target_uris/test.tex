\documentclass{article}
\usepackage{scikgtex}

% ambiguous command

\begin{document}
Lorem ipsum dolor \contribution{Modality}{sit amet}, consectetur adipiscing elit. Pellentesque sed auctor lorem, ut finibus tellus. Nunc faucibus sapien eget eros lobortis tempor. Aenean semper, turpis at lacinia elementum, ante odio fermentum ipsum, vel porta ligula nibh ut dui. Aenean tristique, odio eu pretium accumsan, nibh sapien porta ipsum, eget dictum elit leo eget mauris. Suspendisse condimentum risus eget efficitur pulvinar. Aenean a quam purus. Nulla elit enim, ornare nec eleifend vitae, tincidunt eu neque. Duis dictum metus et eros semper, nec posuere dui facilisis. Suspendisse potenti. Nulla consequat laoreet convallis. Sed sagittis sem massa, non consequat risus egestas maximus.
The role of \researchproblem{antibiotic therapy in managing acute bacterial sinusitis (ABS) in children} is controversial.
The purpose of this study was to determine the \objective{effectiveness of high-dose amoxicillin/potassium clavulanate in the treatment of children diagnosed with ABS}.

This was a \method{randomized, double-blind, placebo-controlled study}.
Children 1 to 10 years of age with a clinical presentation compatible with ABS were eligible for participation.
\method{Patients were stratified according to age (<6 or ≥6 years) and clinical severity and randomly assigned to receive either amoxicillin (90 mg/kg) with potassium clavulanate (6.4 mg/kg) or placebo}.
A symptom survey was performed on days 0, 1, 2, 3, 5, 7, 10, 20, and 30.
Patients were examined on day 14.
Children’s conditions were rated as cured, improved, or failed according to scoring rules.

Two thousand one hundred thirty-five children with respiratory complaints were screened for enrollment; 139 (6.5\%) had ABS.
Fifty-eight patients were enrolled, and 56 were randomly assigned. The mean age was 6630 months.
Fifty (89\%) patients presented with persistent symptoms, and 6 (11\%) presented with nonpersistent symptoms.
In 24 (43\%) children, the illness was classified as mild, whereas in the remaining 32 (57\%) children it was severe.
Of the 28 children who received the antibiotic, 14 (50\%) were cured, 4 (14\%) were improved, 4(14\%) experienced treatment failure, and 6 (21\%) withdrew.
Of the 28children who received placebo, 4 (14\%) were cured, 5 (18\%) improved, and 19 (68\%) experienced treatment failure.
\result{Children receiving the antibiotic were more likely to be cured (50\% vs 14\%) and less likely to have treatment failure (14\% vs 68\%) than children receiving the placebo}.
ABS is a common complication of viral upper respiratory infections. \conclusion{Amoxicillin/potassium clavulanate results in significantly more cures and fewer failures than placebo}, according to parental report of time to resolution.”
\end{document}